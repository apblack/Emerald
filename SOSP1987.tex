\documentstyle[normscite,mypage]{article}
\begin{document}
\bibliographystyle{myalpha}
\newcommand{\assign}{\ifmmode\leftarrow\else$\leftarrow$\fi}

\begin{center}
\LARGE
Fine-Grained Mobility in the Emerald System \\
\large
\vspace{3ex}
Eric Jul, Henry Levy, Norman Hutchinson, and Andrew Black \\
\vspace{3ex}
Department of Computer Science \\
University of Washington \\
Seattle, WA 98195
\end{center}
\normalsize

\begin{abstract}

Emerald is an object-based language and system designed for the construction
of distributed programs.  Emerald allows programmers to take advantage of 
distribution, whilst preserving the advantages of location independence.
In particular, Emerald objects can move
freely from one node to another without their clients needing to be aware
of the move.

We say that Emerald
has {\em fine-grained} mobility because Emerald objects can
be small data objects as well as process objects.  This paper 
is concerned with the efficient implementation of object mobility, with the
impact of mobility on language design, 
and with
techniques for implementing mobility that do not adversely impact the
performance of local operations.  Measurements of the performance of the
current Emerald implementation are included.
\end{abstract}

\baselineskip 4.5ex

\section{Introduction}

Process migration has been implemented or described as a goal of
several distributed systems \cite{accent,butterfield84,powell83,theimer85}.
In these systems, entire address spaces can be moved from node to node.
For example, 
a process manager might 
initiate a move to more evenly share processor load, or users might
initiate remote execution explicitly.  In either case, the
running process is typically
ignorant of its location and unaffected by the move.  

During the last three years, we have designed
and implemented Emerald \cite{em1,em2},  a distributed object-based
language and system.  One of Emerald's design goals was to provide a tool
that would let us
experiment with the use of mobility in distributed
programming.   Mobility in 
Emerald differs from existing process migration schemes
in two important respects:
\begin{itemize}
\item Because Emerald is object-based, the unit of distribution and
mobility is the object.  While some Emerald objects contain processes,
others contain only data: arrays, records and even single integers are all
objects. Thus,
the unit of mobility has a much
smaller grain size than in process migration systems.  Object
mobility in Emerald subsumes both process migration and data transfer.  

\item Mobility plays a critical ro{\circumflex}le in the 
design of the Emerald language.  Not only are there language primitives to
control mobility and location, but certain design choices in conventional
parts of the language (e.g., parameter passing modes) are feasible only
because Emerald objects can move, and move at low cost.

\end{itemize}

The advantages of process migration have been noted in previous work {\cite
somethingorother}.
\begin{enumerate}

\item Load sharing -- By moving objects around the system, one can take
advantage of unused cycles.

\item Communications performance -- Active objects that interact intensively
can be moved to the same node to reduce the communications
cost for the duration of their interaction.

\item Availability -- Objects or object replicas can be moved to different
nodes to provide better failure coverage.

\item Reconfiguration -- Objects can be moved following either a failure
or a recovery, or prior to scheduled down-time.

\item Utilizing special capabilities -- An object can move to take
advantage of unique hardware or software capabilities on a particular
node.
\end{enumerate}
In addition, fine-grained mobility provides three additional benefits:

\begin{enumerate}
\setitemcountto5
\item Data Movement -- Mobility provides a simple way for the programmer
to move data from node to node without having to explicitly package
it.  No separate message passing or file transfer mechanism is required.

\item Invocation Performance -- Mobility has the potential to improve the
performance of remote invocation while retaining call-by-reference
(we discuss this is some detail later).

\item Garbage Collection -- Mobility can help to simplify distributed
garbage collection by moving objects to sites where references 
exist \cite{vestal}.
\end{enumerate}

To our knowledge, Apiary \cite{Hewitt80} is the only other system
intended to support mobility in a style similar to Emerald.

In addition to mobility and distribution, Emerald is intended to
provide efficient execution.  We wanted to demonstrate that
we could achieve performance competitive with standard procedural
languages in the local case and standard remote procedure call
systems in the remote case.  These goals are not trivial
in a location-independent object-oriented
environment.  To meet them, we have
relied heavily on an appropriate choice of language semantics, a tightly-coupled compiler 
and run-time kernel, and careful attention to implementation.

In this paper we concentrate primarily on language and run-time
mechanisms that support fine-grained
mobility while retaining efficient intra-node operation.  First, we 
present an overview of the Emerald language and system, and its
mobility and location primitives.  In doing so, we try to explain why our
attempts to build an efficient object oriented system led us into the realm
of programming language design. Then  
we discuss the implementation of
fine-grained mobility in Emerald and the new problems to which this gives
rise.  Finally, we present
measurements of the implementation and draw conclusions from the
measurements and our design experience.  

\section{Overview of Emerald}

As previously stated, an important goal of Emerald was explicit support
for mobility.  From a conceputal viewpoint, a more important goal
was the existence of a single kind of object.  Object-based systems
typically lie at the ends of the spectrum delineated by 
object-based languages (e.g., Smalltalk and CLU), 
which provide small, local, data objects, and
object-based operating systems (e.g., Hydra and Clouds), 
which provide large, protected, active objects.  Distributed systems such as
Argus and Eden that support both object types have separate object definition
mechanisms for each.  Choosing the right mechanism 
requires that the programmer know ahead of
time all uses to which an object will be put; the alternative is to accept
the inefficiency and inconvenience of using the ``wrong'' mechanism, 
or to reprogram the object
later as needs change.  

The motivation for two distinct definition mechanisms is the need for two
distinct implementations.  On distributed
object-based systems such as Clouds and Eden, 
the local case of the general invocation mechanism can take milliseconds or
tens of milliseconds.  A more restricted and more efficient implementation
is appropriate for objects that are known to be always local; for example,
shared store can be used in preference to messages.

While we beleive in the importance of multiple implementations, we
do not believe that they need to be visible to the programmer.  
In Emerald, a single object definition mechanism with a single semantics 
is used for programming all objects.  This includes small, local,
data-only objects and active, mobile, 
distributed objects.  
However, the Emerald compiler is capable of determining the needs
of each object and generating the appropriate implementation.
Indeed, the compiler can produce
different implementations from the same piece of code, depending
on the context in which it was compiled.
Simple local
objects receive a very light-weight interface, while mobile, distributed
objects receive a heavier interface.   For example,
a tree object used only locally will receive a different
implementation from a tree shared globally.  The various compiler-generated
implementations are described in more detail in section ~\ref{addr}.

The motivation for designing a new langauge, rather than just trying to
apply these ideas to the implementation of an existing language, is that
very often the semantics of a language will preclude efficient
implementation in either the local or remote case.  In designing Emerald,
we constantly kept both inplementations in mind.  Moreover, Emerald has a
unique type system designed to let the programmer state either nothing or a
great deal about the use of a variable; in general, the more information
given to the compiler the better will be the code that it generates.

We believe that
Emerald demonstrates the viability of this
approach and meets our goal of
local performance commensurate with procedural langauges.  
Table ~\ref{localtable} shows the performance of 
several local Emerald operations executed on a 
MicroVAX$^{TM}$ II; reference ~\cite{hutchinson87} contains 
more details of the compiler and its
implementation.  The ``resident global invocation'' time is for a global
object (i.e., one that can move around the network) when invoked
by another object resident on the same node.
By way of comparison, we show the local procedure call time for the
languages C and Concurrent Euclid on the same processor.   Concurrent
Euclid is slower than C because it, like Emerald, must make explicit checks
for stack overflow on each call.

\begin{table}
\centering
\thicklines
\begin{tabular}{||c|c|c||}
\hline
Emerald & Example & Time \\
Operation && (microseconds) \\ \hline\hline
primitive integer invocation & $ i \leftarrow i + 23$ & 0.4 \\ \hline
primitive real invocation & $ x \leftarrow x + 23.0$ & 3.4 \\ \hline
local invocation & localobject.nop & 16.6 \\ \hline
resident global invocation & globalobject.nop & 19.4 \\ \hline
C procedure call & ??? \\ \hline
CE procedure call & ??? \\ \hline
\end{tabular}
\caption{Timings of Local Emerald Invocations}
\label{localtable}
\end{table}

\subsection{Emerald Objects}
\{comment  Hank:  Here I fell asleep.  But please get rid of this list; the
only part of interest is the *optional* process}

Each Emerald object has four components:
\begin{enumerate}
\item A unique network-wide name.

\item A representation, i.e., the data local to the object, which
consists primitive data and references to other objects.

\item A set of operations that can be invoked on the object.

\item An {\em optional} process.
\end{enumerate}
Emerald objects that contain a process are active;  objects without
a process are passive data-only objects that execute only as the result of
invocations.  Multiple invocations of an object may be
active concurrently, and invocations can execute concurrently with the
process.  Synchronization is provided by monitors \cite{hoare74}.

The Emerald language supports the concept of
{\em abstract type} \cite{em1}.  The abstract type of
an object defines its interface, i.e., the number 
of operations that it exports,
their names, and the number and types of
parameters to each operation.   Abstract types permit new object
implementations to be easily added to an existing system.
However, Emerald has no class/instance
hierarchy, in contrast to Smalltalk \cite{smalltalk}. 
Objects are not members of a class; conceptually,
each object carries its own code.  This distinction
is important in a distributed environment where separating an object
from its code would be costly.

Identically implemented Emerald objects on each node do share code.
The code is
stored in a {\em concrete type object}, which implements
one or more abstract types.  Because concrete type
objects are immutable, they can be freely copied.  When an
object is moved to another node, only its data is moved.  If the object
contains a process, part of that data will include the process' stack, but no
code is transferred.  Typically,
we expect that concrete type objects for most mobile objects will already
exist on target nodes.  When a kernel receives an object, it checks 
whether the appropriate
code already exists locally;  if it does not the kernel locates
the concrete type and requests that a copy be moved.


\subsection{Primitives for Mobility}

Object mobility in Emerald is provided by a small set of language
primitives.   An Emerald object can:
\begin{itemize}
\item {\em Locate} an object, i.e., determine on what node it
resides.

\item {\em Move} itself or another object to another location.

\item {\em Fix} an object at a particular node.

\item {\em Unfix} an object, making it mobile again following a fix.

\item {\em Refix} an object, atomically performing an Unfix, Move,
and Fix to a new location.
\end{itemize}
The move primitive is actually a hint;  the kernel is not obliged
to perform the move and the object is not obliged to remain at the
destination site.  Fix and refix have stronger semantics;  if the primitives
succeed the
object will stay at the destination until it is explicitly unfixed.

Central to these primitives is the concept of location, which is
encapsulated in a {\em node} object.  A node object is an abstraction
of a physical machine.  Location may be specified by naming either
a node object or any other object.  If the programmer specifies a
non-node object, the location implied is the node on
which that object resides \cite{black85}.  This 
makes it simple to co-locate objects
without having to discover their locations.

A crucial issue when moving objects containing references
is deciding
how much to move \cite{sollins79}.
An object is part of a graph of references, and one could move a single
object, several levels of objects, or the entire graph.
The simplest approach --- moving the specified object alone --- may
be inappropriate.  Depending on how the object is
implemented, invocations to the moved object may require remote references 
that would have been avoided if other related
objects had been moved as well.

The Emerald programmer may wish to explicitly
specify which objects move together.  For this purpose, the Emerald
language allows the programmer to {\em attach} objects to other
objects.   When a variable is declared, the 
programmer can specify the variable to be an ``attached variable''.  For
example, consider an object {\em queue} that contains the following
statements:
\begin{tabbing}
xxxxx\= \kill
\> {\bf var} {\em aDir} : {\em directory} \\
\> {\bf attached var}  {\em aTree} : {\em balancingTree}

